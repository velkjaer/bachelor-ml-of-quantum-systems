%%%%%%%%%%%%%%%%%%%%%%%%%%%%%%%%%%%%%%%%%%%%%%%%%%%%%
%   Chapter 1: Basic concepts in microfluidics      %
%%%%%%%%%%%%%%%%%%%%%%%%%%%%%%%%%%%%%%%%%%%%%%%%%%%%%




\chapter{Basic concepts in microfluidics}
\chaplab{BasicConcepts}
 \index{lab-on-a-chip systems!basic concepts}

For some references see the papers
Refs.~\cite{Barnkob2010, Augustsson2011, Muller2012, Settnes2012, Karlsen2015}, the conference
proceedings \cite{Augustsson2010, Barnkob2010a, Jensen2013}, and the textbooks by Bruus and Laurell~\cite{Bruus2008, Laurell2014}, respectively, as well as the the web site of the Theoretical Microfluidics Group at DTU Physics~\cite{TMF2015}.

The field of lab-on-a-chip systems  has evolved dramatically since
it was initiated in the early 1990ies. It is a field that is
mainly driven by technological applications, the main vision being
to develop entire bio/chemical laboratories on the surface of
silicon\index{silicon chips} or polymer chips\index{polymer
chips}. Many of the techniques developed the past fifty years in
connection with the revolutionary microelectronics industry can be
used to fabricate lab-on-chip systems. It is, \eg, relatively easy
to etch 100~$\SImum$ wide channels for fluid handling at the
surface of silicon wafers using well-established protocols. But as
we shall see, polymer-based lab-on-a-chip systems have emerged the
recent years, and these systems promise cheaper and faster
production cycles. The study of fluid motion in microsystems is
denoted microfluidics.\index{microfluidics!definition of}



There are several advantages of scaling down standard laboratories
setups by a factor of 1000 or more from the decimeter scale to the
100~$\SImum$ scale. One obvious advantage is the dramatic
reduction in the amount of required sample. A linear reduction by
a factor of $10^3$ amounts to a volume reduction by a factor of
$10^9$, so instead of handling 1~L or 1~mL a lab-on-a-chip system
could easily deal with as little as 1~nL or 1~pL. Such small
volumes allow for very fast analysis, efficient detection schemes,
and analysis even when large amounts of sample are unavailable.
Moreover, the small volumes makes it possible to develop compact
and portable systems that might ease the use of bio/chemical
handling and analysis systems tremendously. Finally, as has been
the case with microelectronics, it is the hope by mass production
to manufacture very cheap lab-on-a-chip systems.

Lab-on-a-chip (LOC) systems\index{LOC, lab-on-a-chip} can be
thought of as the natural generalization of the existing
integrated electronic circuits (IC) and microelectromechanical
systems (MEMS).\index{MEMS, microelectromechanical}
\index{microelectromechanical, MEMS} Why confine the systems to
contain only electric and mechanical parts? Indeed, a lab-on-chip
system can really be thought of a the shrinking of an entire
laboratory to a chip. One example of a system going in that
direction is an integrated lab-on-a-chip system fabricated at DTU Nanotech. This particular system contains optical (lasers and wave
guides), chemical (channels and mixers), and electronic
(photodiodes) components. Perhaps, only our imagination sets the
limits of what could be in a lab-on-a-chip system. It is expected
that lab-on-a-chip systems will have great impact in biotech
industries, farmacology, medical diagnostics, forensics,
environmental monitoring and basic research.

The fundamental laws of Nature underlying our understanding of the
operation of lab-on-a-chip systems are all well-known. We shall
draw on our knowledge from mechanics, fluid dynamics,
electromagnetism, thermodynamics and physical chemistry during
this course. What is new, however, is the interplay between many
different forces and the change of the relative importance of
these forces in the micro-regime as compared to the macro-regime.
Surface effects that often can be neglected at the macro-scale
become increasingly dominant in microfluidics as size is
diminished. For example, it turns out that volume forces like
gravity and inertia that are very prominent in our daily life
become largely unimportant in lab-on-a-chip systems. Instead, we
must get used to the fact that surface related forces, like
surface tension and viscosity, become dominant. As a consequence,
we must rebuild our intuition and be prepared for some surprises
on the way.\index{microfluidics!importance of surface effects}




\section{Fluids and fields}
\seclab{FluidicFields}
\index{fluids and fields, basic concepts}
\index{fields and fluids, basic concepts}
\index{solids, basic concepts}
\index{gas!basic concept}
\index{liquid!basic concept}

The main purpose of a lab-on-a-chip system is to handle fluids. A
fluid, \ie, either a liquid or a gas, is characterized by the
property that it will deform continuously and with ease under the
action of external forces. A fluid does not have a preferred shape
and different parts of it may be rearranged freely without
affecting the macroscopic properties of the fluid. In a fluid the
presence of shear forces, however small in magnitude, will result
in large changes in the relative positions of the fluid elements.
In contrast, the changes in the relative positions of the atoms in
a solid remain small under the action of any small external force.
When applied external forces cease to act on a fluid, it will not
necessarily retract to its initial shape. This property is also in
contrast to a solid, which relaxes to its initial shape when no
longer influenced by external forces.


\subsection{Fluids: liquids and gases}
\seclab{Fluids}
\index{fluids: liquids and gases}

The two main classes of fluids, the liquids and the gases, differ
primarily by the densities and by the degree of interaction
between the constituent molecules as sketched in
\figref{SolLiqGas}. The density $\rho^{{}}_\textrm{gas} \approx
1$~kg$\:$m$^{-3}$\index{gas!density, typical value} of an ideal
gas is so low, at least a factor of $10^3$ smaller than that of a
solid, that the molecules move largely as free particles that only
interact by direct collisions at atomic distances, $\approx
0.1~\SInm$. The relatively large distance between the gas
molecules, $\approx 3~\SInm$, makes the gas compressible. The
density $\rho^{{}}_\textrm{liq} \approx 10^3$~kg$\:$m$^{-3}$ of a
liquid is comparable to that of a solid, \ie, the molecules are
packed as densely as possible with a typical average
inter-molecular distance of 0.3~nm, and a liquid can for many
practical purposes be considered incompressible.
\index{water!density at 20$^\circ$C}
\index{density of water at 20$^\circ$C}

\begin{figure}
\centerline{
  \includegraphics[]{figs/chap01/SolLiqGas.eps}}
\caption[Molecular picture of solids, liquids and gases]
{\figlab{SolLiqGas} (a) A sketch of a typical solid with 0.1~nm
wide molecules (atoms) and a lattice constant of 0.3~nm. The atoms
oscillate around the indicated equilibrium points forming a
regular lattice. (b) A sketch of a liquid with the same molecules
and same average inter-molecular distance 0.3~nm as in panel (a).
The atoms move around in an thermally induced irregular pattern.
(c) A sketch of a gas with the same atoms as in panel (a). The
average inter-atomic distance is 3~nm, and the motion is free
between the frequent inter-atomic collisions.}
\end{figure}

The inter-molecular forces in a liquid are of quite intricate
quantum and electric nature since each molecule is always
surrounded by a number of molecules within atomic distances
\index{inter-molecular forces}. In model calculations of simple
liquids many features can be reproduced by assuming the simple
Lennard--Jones pair-interaction potential, $V^{{}}_\textrm{LJ}(r)
= 4\ve \big[(\sigma/r)^{12} - (\sigma/r)^6\big]$, between any pair
of molecules\index{Lennard--Jones potential}. Here $r$ is the
distance betweens the molecules, while the maximal energy of
attraction $\ve$ and the collision diameter $\sigma$ are material
parameters typical of the order $100$~K$\times\kB$ and
$0.3~\SInm$, respectively. The corresponding inter-molecular force
is given by the derivative $F^{{}}_\textrm{LJ}(r) =
-dV^{{}}_\textrm{LJ}/dr$. The Lennard--Jones potential is shown in
\figref{LiqMolecules}a and discussed further in
\exref{LennardJones}.

At short time intervals and up to a few molecular diameters the
molecules in a liquid are ordered almost as in a solid. However,
whereas the ordering in solids remains fixed in time and
space,\footnote{the molecules in a solid execute only small,
thermal oscillations around equilibrium points well-described by a
regular lattice} the ordering in liquids fluctuates. In some sense
the thermal fluctuations are strong enough to overcome the
tendency to order, and this is the origin of the ability of
liquids to flow\index{thermal fluctuations!molecular disorder in
liquids}.


\subsection{The continuum hypothesis and fluid particles}
\seclab{ContHyp}
\index{continuum hypothesis}
\index{fluid particles!definition}

Although fluids are quantized on the length scale of
inter-molecular distances (of the order 0.3~nm for liquids and
3~nm for gases), they appear continuous in most lab-on-a-chip
applications, since these typically are defined on macroscopic
length scales of the order 10~$\SImum$ or more. In this course we
shall therefore assume the validity of the continuum hypothesis,
which states that the macroscopic properties of a fluid is the
same if the fluid were perfectly continuous in structure instead
of, as in reality, consisting of molecules. Physical quantities
such as the mass, momentum and energy associated with a small
volume of fluid containing a sufficiently large number of
molecules are to be taken as the sum of the corresponding
quantities for the molecules in the volume.

\begin{figure}
\centerline{
  \includegraphics[]{figs/chap01/LiqMolecules.eps}}
\caption[Lennard--Jones potential]{\figlab{LiqMolecules} (a) The
Lennard--Jones pair-potential $V^{{}}_\textrm{LJ}(r)$ often used
to describe the interaction potential between two molecules at
distance $r$, see also \exref{LennardJones}. For small distances,
$r<r^{{}}_0\approx0.3~\SInm$ the interaction forces are strongly
repulsive (gray region), while for large distances, $r>r^{{}}_0$,
they are weakly attractive. (b) A sketch of some measured physical
quantity of a liquid as a function of the volume
$\vol^{{}}_\textrm{probe}$ probed by some instrument. For
microscopic probe volumes (left gray region) large molecular
fluctuations will be observed. For mesoscopic probe volumes (white
region) a well-defined local value of the property can be
measured. For macroscopic probe volumes (right gray region) gentle
variations in the fluid due to external forces can be observed.}
\end{figure}

The continuum hypothesis leads to the concept of fluid particles,
the basic constituents in the theory of fluids. In contrast to an
ideal point-particle in ordinary mechanics, a fluid particle in
fluid mechanics has a finite size. But how big is it? Well, the
answer to this question is not  straight forward. Imagine, as
illustrated in \figref{LiqMolecules}b, that we probe a given
physical quantity of a fluid with some probe sampling a volume
$\vol^{{}}_\textrm{probe}$ of the fluid at each measurement. Let
$\vol^{{}}_\textrm{probe}$ change from (sub-)atomic to macroscopic
dimensions. At the atomic scale (using, say, a modern AFM or STM)
we would encounter large fluctuations due to the molecular
structure of the fluid, but as the probe volume increases we soon
enter a size where steady and reproducible measurements are
obtained. This happens once the probe volume is big enough to
contain a sufficiently large number of molecules, such that
well-defined average values with small statistical fluctuations
are obtained. As studied in \exref{FluidParticle} a typical
possible side length $\lambda^*$ in a cubic fluid particle is
%
 \beq{FluidParticle} \lambda^* \approx 10~\textrm{nm}.
 \eeq
%
Such a fluid particle contains approximately $4\times10^4$
molecules and exhibits number fluctuations of the order 0.5\%.
\index{thermal fluctuations!number of molecules} If the size of
the fluid particle is taken too big the probe volume could begin
to sample regions of the fluid with variations in the physical
properties due to external forces. In that case we are beyond the
concept of a constituent particle and enters the regime we
actually would like to study, namely, how do the fluid particles
behave in the presence of external forces.

A fluid particle must thus be ascribed a size $\lambda^*$ in the
mesoscopic range.\index{mesoscopic regime} It must be larger than
microscopic lengths ($\simeq 0.3~\SInm$) to contain a sufficiently
large amount of molecules, and it must be smaller than macroscopic
lengths ($\simeq 10~\SImum$) over which external forces change the
property of the fluid. Of course, this does not define an exact
size, and in fluid mechanics it is therefore natural to work
physical properties per volume, such as mass density,\index{mass
density} energy density,\index{energy density} force
density\index{force density} and momentum density\index{momentum
density}. In such considerations the volume is taken to the limit
of a small, but finite, fluid particle volume, and not to the
limit of an infinitesimal volume.

The continuum hypothesis breaks down when the system under
consideration approaches molecular scale. This happens in
nanofluidics, \eg, in liquid transport through nano-pores in cell
membranes or in artificially made nano-channels.

\subsection{The velocity, pressure and density field}
\seclab{VelocityField}
\index{velocity field!basic concept}
\index{pressure field!basic concept}
\index{density field!basic concept}


Once the concept of fluid particles in a continuous fluid has been
established we can move on and describe the physical properties of
the fluid in terms of fields. This can basically be done in two
ways as illustrated in \figref{VelocityField} for the case of the
velocity field. In these notes we shall use the Eulerian
description, \figref{VelocityField}a,\index{Eulerian description
of velocity fields} where one focuses on fixed points $\rrr$ in
space and observe how the fields evolve in time at these points,
\ie, the position $\rrr$ and the time $t$ are independent
variables. The alternative is the Lagrangian description,
\index{Lagrangian description of velocity fields}
\figref{VelocityField}b, where one follows the history of
individual fluid particles as the move through the system, \ie,
the coordinate $\rrr^{{}}_a(t)$ of particle $a$ depends on time.

In the Eulerian description the value of any field variable
$F(\rrr,t)$ is defined as the average value of the corresponding
molecular quantity $F^{{}}_\textrm{mol}(\rrr',t)$ for all the
molecules contained in some liquid particle of volume $\Delta
\vol(\rrr)$ positioned at $\rrr$ at time $t$,
%
 \beq{Fmol}
 F(\rrr,t) = \big\langle F^{{}}_\textrm{mol}(\rrr',t)
 \big\rangle^{{}}_ {\rrr' \in \Delta \vol(\rrr)}.
 \eeq
%
The field variables can be scalars (such as density $\rho$,
viscosity $\eta$, pressure $p$, temperature $T$, and free energy
$\mathcal{F}$),\index{scalar fields} vectors (such as velocity
$\vvv$, current density $\JJJ$, pressure gradient $\nablabf p$,
force densities $\fff$, and electric fields $\EEE$)\index{vector
field!basic concept} and tensors (such as stress tensor $\sigma$
and velocity gradient $\nablabf \vvv$)\index{tensor field!basic
concept}.

To obtain a complete description of the state of a moving fluid it
is necessary to know the three components of the velocity field
$\vvv(\rrr,t)$ and any two of the thermodynamical variables of the
fluid, \eg, the pressure field $p(\rrr,t)$ and the density field
$\rho(\rrr,t)$. All other thermodynamical quantities can be
derived from these fields together with the equation of state of
the fluid.

\begin{figure}
\centerline{
  \includegraphics[width=\textwidth]{figs/chap01/VelocityField.eps}}
\caption[Velocity fields: Eulerian and
Lagrangian]{\figlab{VelocityField} (a) The velocity field
$\vvv(\rrr,t)$ in the Eulerian description at the point $\rrr$ at
the two times $t\sm\Delta t$ and $t$. The spatial coordinates
$\rrr$ are independent of the temporal coordinate $t$. (b) The
Lagrangian velocity fields $\vvv\big(\rrr^{{}}_a(t),t\big)$ and
$\vvv\big(\rrr^{{}}_b(t),t\big)$ of fluid particles $a$ (white)
and $b$ (dark gray). The particles pass the point $\rrr$ at time
$t\sm\Delta t$ and $t$, respectively. The particle coordinates
$\rrr^{{}}_{a,b}(t)$ depend on $t$. Note that
$\rrr^{{}}_a(t\sm\Delta t) = \rrr$ and $\rrr^{{}}_b(t) = \rrr$.}
\end{figure}

\section{SI units and mathematical notation}
\seclab{SIunitsMathNotation}

Notation is an important part in communicating scientific and
technical material. Especially in fluid mechanics the mathematical
notation is important due to the involved many-variable
differential calculus on the scalar, vector and tensor fields
mentioned in the previous section. Instead of regarding units and
notation as an annoying burden the student should instead regard
it as part of the trade that need to be mastered by the true
professional. Learn the basic rules, and stick to them thereafter.


\subsection{SI units}
\seclab{SIunits}
\index{SI units}

Throughout these notes we shall use the SI units. If not truly
familiar with this system, the name and spelling of the units, and
the current best values of the fundamental physical constants of
Nature the reader is urged to consult the web-site of National
Institute of Standards and Technology (NIST) for constants, units,
and uncertainty at\index{NIST}\index{units}
%
 \beq{NIST} \textrm{http:/$\!$/physics.nist.gov/cuu/} \; .
 \eeq
%
A scalar physical variable is given by a number of significant
digits, a power of ten and a proper SI unit. The power of ten can
be moved to the unit using the prefixes (giga, kilo, micro, atto
etc.). The SI unit can be written in terms of the seven
fundamental units or using the derived units. As an example the
viscosity $\eta$ of water at $20^\circ$C is written as
\index{water!viscosity at 20$^\circ$C}
\index{viscosity of water at 20$^\circ$C}
%
 \beq{SIunitExample}
 \eta =
 1.002 \times 10^{-3}~\SIkg\: \SIm^{-1} \SIs^{-1}
 = 1.002~\SImPas.
 \eeq\index{notation!numbers with SI units}
%
Note the multiplication sign before the power of ten and the space
after it, and note that the SI units are written in roman and
\textit{not in italics}. Most type setting systems will
automatically use italics for letters written in equations. Note
also the use of space and no multiplication signs between the
units. Be aware that even though many units are capitalized as are
the names of the physicists given rise to them, \eg, Pa and
Pascal, the unit itself is never capitalized when written in full,
\eg, pascal. Also, the unit is written pascal without plural form
whether there is one, five or 3.14 of them.

There will be two exceptions from the strict use of SI units.
Sometimes, just as above, temperatures will be given in $^\circ$C,
so be careful when inserting values for temperature in formulae.
Normally, a temperature $T$ in an expression calls for values in
kelvin. The other exception from SI units is the atomic unit of
energy, electronvolt (eV),
%
 \beq{eV} 1~\textrm{eV} = 1.602\times10^{-19}~\textrm{J}
 = 0.1602~\textrm{aJ}.
 \eeq
%
Note, that it would be possible to use attojoule instead of
electronvolt, but this is rarely done.


\subsection{Vectors, derivatives and the index notation}
\seclab{MathNotation}
\index{notation}
\index{index notation}
\index{vector field!index notation}
\index{tensor field!index notation}
\index{Einstein summation convention}

The mathematical treatment of microfluidic problems is complicated
due to the presence of several scalar, vector and tensor fields
and the non-linear partial differential equations that govern
them. To facilitate the treatment some simplifying notation is
called for.

First, a suitable coordinate system must be chosen. We shall
encounter three: Cartesian coordinates $(x,y,z)$ with
corresponding unit vectors $\een_x$, $\een_y$, and $\een_z$;
cylindrical coordinates $(r,\phi,z)$ with corresponding unit
vectors $\een_r$, $\een_\phi$, and $\een_z$; and spherical
coordinates $(r,\theta,\phi)$  with corresponding unit vectors
$\een_r$, $\een_\theta$, and $\een_\phi$. The Cartesian unit
vectors are special since they are constant in space, whereas all
other sets of unit vectors depend on position in space. For
simplicity, we postpone the usage of the curvilinear coordinates
to later chapters and use only Cartesian coordinates in the
following.

The position vector $\rrr = (r^{{}}_x, r^{{}}_y, r^{{}}_z) =
(x,y,z)$ can be written as
%
 \beq{rvector} \rrr =
 r^{{}}_x\: \een_x + r^{{}}_y\: \een_y + r^{{}}_z\: \een_z =
 x\: \een_x + y\: \een_y + z\: \een_z.
 \eeq
%
In fact, any vector $\vvv$ can be written in terms of its
components $v^{{}}_i$ (where for cartesian coordinates $i=x,y,z$)
as
%
 \beq{vector} \vvv = \sum_{i=x,y,z} v^{{}}_i\: \een_i
 \equiv v^{{}}_i\: \een_i
 \eeq
%
In the last equality we have introduced the Einstein summation
convention: per definition a repated index alway implies a
summation over that index. Other examples of this handy notation,
the so-called index notation, is the scalar product,\index{scalar
product of vectors}\index{vector scalar product}
%
 \beq{ScalarProd}
 \vvv \cdot \uuu = v^{{}}_i u^{{}}_i,
 \eeq
%
the length $v$ of a vector $\vvv$,
%
 \beq{LengthVec}
 v = |\vvv| = \sqrt{\vvv^2} =
 \sqrt{\vvv\cdot\vvv} = \sqrt{v^{{}}_i v^{{}}_i},
 \eeq
%
and the $i$th component of the vector-matrix equation $\uuu = M
\vvv$,
%
 \beq{MatrixProd}
 u^{{}}_i = M^{{}}_{ij}\:v^{{}}_j.
 \eeq
%
Further studies of the index notation can be found in
\exref{IndexNotation}.

For the partial derivatives\index{partial derivative, notation}
\index{derivative, notation} of some function $F(\rrr,t)$ we use
the symbols $\pp_i$, with $i = x,y,z$, and $\pp_t$,
\index{time-derivative!partial $\pp_t$}
%
 \beq{PartDeriv}
 \pp_x F \equiv \frac{\pp F}{\pp x},
 \quad \textrm{and} \quad
 \pp_t F \equiv \frac{\pp F}{\pp t},
 \eeq
%
while for the total time-derivative, as, \eg, in the case of the
Lagrangian description of some variable $F\big(\rrr(t),t\big)$
following the fluid particles (see \figref{VelocityField}b), we
use the symbol $d^{{}}_t$,\index{time-derivative!total
$d^{{}}_t$}\index{total time-derivative $d^{{}}_t$}
%
 \beq{TotalDeriv}
 d^{{}}_t F \equiv \frac{d F}{d t} =
 \pp_t F  + \big(\pp_t r^{{}}_i\big) \pp_i F =
 \pp_t F + v^{{}}_i\pp_iF.
 \eeq
%
The nabla operator $\nablabf$ containing the spatial derivatives
plays an important role in differential calculus. In Cartesian
coordinates it is given by\index{nabla operator, definition}
%
 \beq{nabla}
 \nablabf \equiv \een_x\pp_x + \een_y\pp_y +  \een_z\pp_z
 =  \een_i \pp_i.
 \eeq
%
Note that we have written the differential operators to the right
of the unit vectors. While not important in Cartesian coordinates
it is crucial when working with curvilinear coordinates. The
Laplace operator, which appears in numerous partial differential
equations in theoretical physics, is just the square of the nabla
operator,\index{Laplace operator!Cartesian coordinates}
%
 \beq{LaplaceOp}
 \nablabf^2 \equiv \pp_i \pp_i.
 \eeq
%
In terms of the nabla-operator the total time derivative in
\eqref{TotalDeriv} can be written as
%
 \beq{TotalDeriv2}
 d^{{}}_t F\big(\rrr(t),t\big) =
 \pp_t F + (\vvv\cdot\nabla)F.
 \eeq
%

When working with vectors and tensors it is advantageous to use
the following two special symbols: the Kronecker delta
$\delta^{{}}_{ij}$,\index{Kronecker delta $\delta^{{}}_{ij}$}
%
 \beq{KroneckerDelta}
 \delta^{{}}_{ij} = \left\{ \begin{array}{l@{}l}
 1,&\quad \textrm{for $i=j$},\\
 0,&\quad \textrm{for $i\neq j$}, \end{array} \right.
 \eeq
%
and the Levi--Civita symbol $\epsilon^{{}}_{ijk}$,
\index{Levi--Civita symbol $\epsilon^{{}}_{ijk}$}
%
 \beq{LeviCivita}
 \epsilon^{{}}_{ijk} =
 \left\{ \begin{array}{r@{}l}
 +1,\quad &\textrm{if $(ijk)$ is \makebox[7ex][c]{an even}
 permutation of $(123)$ or $(xyz)$},\\
 -1,\quad &\textrm{if $(ijk)$ is \makebox[7ex][c]{an odd}
 permutation of $(123)$ or $(xyz)$},\\
 0,\quad  &\textrm{otherwise}. \end{array} \right.
 \eeq
%
In the index notation the Levi--Cevita symbol appears directly in
the definition of the cross product of two vectors $\uuu$ and
$\vvv$,\index{cross product of vectors}\index{vector cross
product}
%
 \beq{CrossProd}
 (\uuu \times \vvv)^{{}}_i \equiv
 \epsilon^{{}}_{ijk} u^{{}}_j v^{{}}_k.
 \eeq
%
and in the definition of the rotation $\nabla \times \vvv$ of a
vector $\vvv$. The expression for the $i$th component of the
rotation is:
%
 \beq{RotationDef}
 (\nabla \times \vvv)^{{}}_i \equiv
 \epsilon^{{}}_{ijk} \pp_j v^{{}}_k.
 \eeq
%
To calculate in the index notation the rotation of a rotation,
such as $\nablabf \times \nablabf \times \vvv$, or the rotation of
a cross product it is very helpful to know the following
expression for the product of two Levi--Civita symbols with one
pair of repeated indices (here $k$):
%
 \beq{TwoLeviCivita}
 \epsilon^{{}}_{ijk} \epsilon^{{}}_{lmk} =
 \delta^{{}}_{il} \delta^{{}}_{jm} -
 \delta^{{}}_{im} \delta^{{}}_{jl}.
 \eeq
%
Note the plus sign when pairing index 1 with 1 and 2 with 2
(direct pairing), while a minus sign appears when pairing index 1
with 2 and 2 with 1 (exchange pairing).


\section{The continuity equation}
\seclab{ContEq}
\index{continuity equation!mass}

We have now cleared the ground for the derivation of our first
fundamental equation of fluid mechanics, the continuity equation.
This equation expresses the conservation of mass in classical
mechanics.

\begin{figure}
\centerline{
  \includegraphics[height=40mm]{figs/chap01/ContEq.eps}}
\caption[Continuity equation]{\figlab{ContEq} A sketch of the
current density field $\rho \vvv$ flowing through an arbitrarily
shaped region $\Omega$. Any infinitesimal area $da$ is associated
with an outward pointing unit vector $\nnn$ perpendicular to the
local surface. The current through the area $da$ is given by $da$
times the projection $\rho\vvv\cdot\nnn$ of the current density on
the surface unit vector.\index{surface unit vector}\index{unit
vector of a surface}}
\end{figure}


\subsection{Compressible fluids}
\seclab{ContEqCompress}
\index{compressible fluids}

We begin by considering the general case of a compressible fluid,
\ie, a fluid where the density $\rho$ may vary as function of
space and time. Consider an arbitrarily shaped, but fixed, region
$\Omega$ in the fluid as sketched in \figref{ContEq}. The total
mass $M(\Omega,t)$ inside $\Omega$ can be expressed as a volume
integral over the density $\rho$,
%
 \beq{MassOmega}
 M(\Omega,t) = \int_\Omega d\rrr\: \rho(\rrr,t),
 \eeq
%
where we have written the infinitesimal integration volume as
$d\rrr$. Since mass can neither appear nor disappear spontaneously
in non-relativistic mechanics, $M(\Omega,t)$ can only vary if mass
is flowing into or out from the region $\Omega$ through its
surface $\pp \Omega$. The mass current density $\JJJ$ in any point
in space is given by
%
 \beq{MassCurrentDensity}
 \JJJ(\rrr,t) = \rho(\rrr,t)\: \vvv(\rrr,t),
 \eeq
%
where $\vvv$ is the Eulerian velocity field.

Since the region $\Omega$ is fixed the time-derivative of the mass
$M(\Omega,t)$ can be calculated either as a volume integral by
using \eqref{MassOmega},
%
 \beq{dtMass1}
 \pp_tM(\Omega,t) =
 \pp_t \int_\Omega d\rrr\: \rho(\rrr,t) =
 \int_\Omega d\rrr\: \pp_t\rho(\rrr,t),
 \eeq
%
or as a surface integral over $\pp \Omega$ of the mass current
density using \eqref{MassCurrentDensity} and \figref{ContEq},
%
 \beq{dtMass2}
 \pp_tM(\Omega,t) =
 - \int_{\pp\Omega} da\: \nnn\scap\Big(\rho(\rrr,t)\vvv(\rrr,t)\Big) =
 - \int_\Omega d\rrr\: \nabla\scap\Big(\rho(\rrr,t)\vvv(\rrr,t)\Big).
 \eeq
%
The last expression is obtained by applying Gauss's theorem
\index{Gauss's theorem}. The minus sign is there since the mass
inside $\Omega$ diminishes if $\rho\vvv$ is parallel to the
outward pointing surface vector $\nnn$. From
\eqsref{dtMass1}{dtMass2} it follows immediately that
%
 \beq{dtMass}
 \int_\Omega d\rrr\: \bigg[ \pp_t\rho(\rrr,t) +
 \nabla\scap\big(\rho(\rrr,t)\vvv(\rrr,t)\Big)\bigg] = 0.
 \eeq
%
This results is true for any choice of region $\Omega$. But this
is only possible if the integrand is zero. Thus we have derived
the continuity equation,
%
 \beq{ContEqCompress}
 \pp_t\rho + \nabla\scap\big(\rho \vvv \big) = 0
 \quad \textrm{or} \quad
 \pp_t\rho + \nabla\scap\JJJ = 0
 \eeq
%
Note that since also electric charge is a conserved quantity, the
argument holds if $\rho$ is substituted by the charge density
$\rho^{{}}_\textrm{el}$, and \eqref{ContEqCompress} can be read as
the continuity equation for charge instead as for mass.
\index{continuity equation!charge}

\subsection{Incompressible fluids}
\seclab{ContEqIncompress}
\index{incompressible fluids}

In many cases, especially in microfluidics, where the flow
velocities are much smaller than the sound velocity in the liquid,
the fluid can be treated as being incompressible. This means that
$\rho$ is constant in space and time, and the continuity equation
(\ref{eq:ContEqCompress}) is simplified to the following form,
%
 \beq{ContEqIncompress}
 \nabla\scap\vvv  = 0
 \quad \textrm{or} \quad
 \pp_i v^{{}}_i = 0,
 \eeq
%
a result we shall use extensively in this course.


\section{The Navier--Stokes equation}
\seclab{NSeq}
\index{Navier--Stokes equation!derivation}
\index{Newton's second law}
\index{forces on fluid particles}
\index{fluid particles!forces on}

Newton's second law for fluid particles is called the
Navier--Stokes equation. It constitutes the equation of motion for
the Eulerian velocity field $\vvv(\rrr,t)$. For an ordinary
particle of mass $m$ influenced by external forces $\sum_j
\FFF^{{}}_j$ Newton's second law reads
%
 \beq{Newton2nd}
 m\: d^{{}}_t \vvv  = \sum_j \FFF^{{}}_j.
 \eeq
%
In fluid mechanics, as discussed in \secref{VelocityField}, we
divide by the volume of the fluid particle and thus work with the
density $\rho$ and the force densities $\fff^{{}}_j$. Moreover, in
fluid mechanics we must be careful with the time-derivative of the
velocity field $\vvv$. As illustrated in \figref{VelocityField}
the Eulerian velocity field $\vvv(\rrr,t)$ is not the velocity of
any particular fluid particle, as it should be in Newton's second
law \eqref{Newton2nd}. To obtain a physically correct equation of
motion a special time-derivative, the so-called material
time-derivative $D^{{}}_t$ defined in the following subsection, is
introduced for Eulerian velocity fields. Our first version of the
Navier--Stokes equation thus takes the form
%
 \beq{NS1}
 \rho\: D^{{}}_t \vvv  = \sum_j \fff^{{}}_j.
 \eeq
%
In the following we derive explicit expressions for the material
time-derivative $D^{{}}_t$ and various force densities
$\fff^{{}}_j$.


\subsection{The material time-derivative}
\seclab{MaterialDeriv}
\index{material (substantial) time-derivative $D^{{}}_t$}
\index{substantial (material) time-derivative $D^{{}}_t$}
\index{time-derivative!material (substantial) $D^{{}}_t$}

The material (or substantial) time-derivative is the one obtained
when following the flow of a particle, \ie, when adopting a
Lagrangian description. We have already in \eqref{TotalDeriv2}
found the appropriate expression, so using that on the velocity
field $\vvv$ we arrive at
%
 \beq{MatDeriv1}
 \rho\: D^{{}}_t \vvv(\rrr,t) \equiv
 \rho\: d^{{}}_t \vvv\big(\rrr(t),t\big)
 = \pp_t\vvv(\rrr,t) + (\vvv\scap\nablabf)\vvv(\rrr,t).
 \eeq
%
Note the use of the Lagrangian velocity field in the definition.

We can derive the same result by first noting that the total
differential of the Eulerian velocity field in general is given by
$d\vvv = dt \pp_t\vvv  + (d\rrr\cdot\nablabf)\vvv$. Second, if we
insist on calculating the change due to the flow of a particular
fluid particle we must have $d\rrr = \vvv dt$. Combining these two
expression leads to \eqref{MatDeriv1}. The same analysis applies
for any flow variable, and we can conclude that the material
time-derivative $D^{{}}_t$ is given by
\index{notation!time-derivative}
%
 \beq{MatDeriv2}
 D^{{}}_t = \pp_t + (\vvv\scap\nablabf).
 \eeq
%
The Navier--Stokes equation now takes the form
%
 \beq{NS2}
 \rho\Big(\pp_t \vvv + (\vvv\scap\nablabf)\vvv\Big) =
 \sum_j \fff^{{}}_j,
 \eeq
%
and we proceed by finding the expressions for the force densities
$\fff^{{}}_j$.


\subsection{Body forces}
\seclab{BodyForces}
\index{body force, definition}

The body forces are external forces that act throughout the entire
body of the fluid. In this course we shall in particular work with
the gravitational force (in terms of the density $\rho$ and the
acceleration of gravity $\ggn$) and the electrical force (in terms
of the charge density $\rho^{{}}_\textrm{el}$ of the fluid and the
external electric field $\EEE$). The resulting force density from
these two body forces is
%
 \beq{BodyForce}
\index{gravitational body force}
\index{electrical body force}
 \fff^{{}}_\textrm{grav} + \fff^{{}}_\textrm{el} =
 \rho \ggn + \rho^{{}}_\textrm{el} \EEE.
 \eeq

\subsection{The pressure-gradient force}
\seclab{PressureGradient}
\index{pressure-gradient force}

Consider a region $\Omega$ in a fluid with a surface $\pp \Omega$
with a surface normal vector $\nnn$. The total external force
$\FFF^{{}}_\textrm{pres}$ acting on this region due to the
pressure $p$ is given by the surface integral of $-\nnn p$,
%
 \beq{PressureForce}
 \FFF^{{}}_\textrm{pres} =
 \int_{\pp \Omega} da\: (-\nnn p) =
 \int_{\pp \Omega} da\: \nnn (-p) =
 \int_{\Omega} d\rrr\: (-\nabla p).
 \eeq
%
The minus sign is necessary since $\nnn p$ is the outward force
per area from the region acting on the surroundings, and not the
other way around. In the last step of \eqref{PressureForce} the
surface integral is converted to a volume integral using Gauss's
theorem.\footnote{The $i$th component of \eqref{PressureForce} is
found by Gauss's theorem using the vector field $-p
\een_i$:\\
\centerline{$\een_i\scap\int_{\pp \Omega} da\: \nnn (-p) =
\int_{\pp \Omega} da\: \nnn\scap(-p \een_i) = \int_{\Omega}
d\rrr\: \nabla\scap(-p\een_i) = \een_i\scap\int_{\Omega} d\rrr\:
(-\nabla p)$.\index{Gauss's theorem}}} The integrand of the
volume integral can thus be identified as the force density due to
the pressure:
%
 \beq{PresGradForce}
 \fff^{{}}_\textrm{pres} = - \nabla p.
 \eeq
%


\subsection{The viscous force and the viscous stress tensor}
\seclab{ViscousStressTensor}
\index{viscous!force}
\index{viscous!stress tensor $\sigma^{\prime}_{ik}$}

Consider the same region $\Omega$ of the fluid as in the previous
subsection. Due to the viscous nature of the fluid, $\Omega$ will
be subject to frictional forces on its surface $\pp \Omega$ from
the flow of the surrounding liquid.\index{frictional force} The
frictional force $d\FFF$ on a surface element $da$ with the normal
vector $\nnn$ must be characterized by a tensor rank two since two
vectors are needed to determine it: the force and the surface
normal need not point in the same direction. This tensor is
denoted the viscous stress tensor $\sigma^{\prime}_{ik}$, and it
expresses the $i$th component of the friction force per area
acting on a surface element oriented with the surface normal
parallel to the $k$th unit vector $\een_k$. Thus
%
 \beq{FsigmaN}
 dF_i = \sigma^{\prime}_{ik} n^{{}}_k\: da.
 \eeq
%
The internal friction is only non-zero when fluid particles move
relative to each other, hence $\sigma^{\prime}$ depends only on
the spatial derivatives of the velocity. For the small velocity
gradients encountered in microfluidics we can safely assume that
only first order derivatives enter the expression for $\sigma'$,
thus $\sigma^{\prime}_{ik}$ must depende linearly on the velocity
gradients $\pp_i v^{{}}_k$.

We can pinpoint the expression for $\sigma^{\prime}_{ik}$ further
by noticing that it must vanish when the liquid is rotating as a
whole, \ie, when the velocity field has the form $\vvv =
\bm{\omega}\times\rrr$, where $\bm{\omega}$ is an angular velocity
vector. For this velocity field we have the anti-symmetric
relation $\pp_k v^{{}}_i = -\pp_i v^{{}}_k$, so $\sigma^{\prime}$
vanishes if it only contains the symmetric combinations $\pp_k
v^{{}}_i + \pp_i v^{{}}_k$ and $\pp_j v^{{}}_j$ of the first order
derivatives. The most general tensor of rank two satisfying these
conditions is\index{tensor field!viscous stress tensor}
%
 \beq{ViscousStressTensor}
 \sigma^{\prime}_{ik} =
 \eta\: \Big(\pp_k v^{{}}_i + \pp_i v^{{}}_k -
 \frac{2}{3}\delta^{{}}_{ik}\pp_jv^{{}}_j\Big) +
 \zeta\: \delta^{{}}_{ik}\pp_jv^{{}}_j.
 \eeq
%
The coefficients $\eta$ and $\zeta$ are denoted the viscosity and
second viscosity, respectively. Note, that the viscous stress
tensor in \eqref{ViscousStressTensor} has been normalized such
that the term with the prefactor $\eta$ has zero trace.

In analogy with the pressure force, the viscous force
$\FFF^{{}}_\textrm{visc}$ can be written as a surface integral,
which by use of Gauss's theorem is converted into a volume
integral,
%
 \beq{ViscForce}
 \big(\FFF^{{}}_\textrm{visc}\big)^{{}}_i =
 \int_{\pp \Omega} da\:  n^{{}}_k \sigma^{\prime}_{ik} =
 \int_{\Omega} d\rrr\: \pp_k \sigma^{\prime}_{ik}.
 \eeq
%
The integrand is simply the $i$th component of the viscous force
density $\fff^{{}}_\textrm{visc}$,
%
 \beq{ViscGradForce}
 \big(\fff^{{}}_\textrm{visc}\big)^{{}}_i =
 \pp_k \sigma^{\prime}_{ik}.
 \eeq


\subsection{The Navier--Stokes equation for compressible fluids}
\seclab{NScomp}
\index{Navier--Stokes equation!compressible fluids}

It is customary to combine the force densities due to pressure and
viscosity since they both are expressed as gradients. The stress
tensor $\sigman_{ik}$ is defined as\index{stress tensor
$\sigma^{}_{ik}$}
%
 \beq{StressTensor}
 \sigman_{ik} \equiv
 -p\:\deltan_{ik} + \sigma^{\prime}_{ik},
 \eeq
%
which then allow us to write
%
 \beq{StressTensorForce}
 \big(\fff^{{}}_\textrm{pres} +
 \fff^{{}}_\textrm{visc}\big)^{{}}_i =
 \pp_k \sigman_{ik}.
 \eeq
%

Inserting the force density expressions
\eqsref{BodyForce}{StressTensorForce} into \eqref{NS2} we obtain
the full Navier--Stokes equation for compressible fluids:
%
 \beq{NS3}
 \rho\Big(\pp_t v^{{}}_i + (v^{{}}_j\pp_j)v^{{}}_i\Big) =
 \pp_k \sigman_{ik} +
 \rho g^{{}}_i + \rho^{{}}_\textrm{el} E^{{}}_i.
 \eeq



\subsection{The Navier--Stokes equation for incompressible fluids}
\seclab{NSincomp}
\index{Navier--Stokes equation!incompressible fluids}

In case of incompressible fluids the continuity equation is valid
in its simple form $\pp_k v^{{}}_k = 0$. This reduces the stress
tensor \eqref{ViscousStressTensor} to
%
 \beq{ViscousStressTensorInc}
 \sigma^{\prime}_{ik} =
 \eta\: \Big(\pp_k v^{{}}_i + \pp_i v^{{}}_k \Big).
 \eeq
%
If furthermore the viscosity $\eta$ is constant, the divergence of
the stress tensor is simply
%
 \beq{DivStressTensor}
 \pp_k \sigman_{ik} = -\pp_i p + \eta\pp_k\pp_k v^{{}}_i.
 \eeq
%
The resulting form of the Navier--Stokes equation is the one we
shall use in this course,
%
 \beq{NSincomp}
 \rho\Big(\pp_t \vvv + (\vvv\scap\nablabf)\vvv\Big) =
 -\nablabf p + \eta\nablabf^2 \vvv +
 \rho\: \ggn + \rho^{{}}_\textrm{el} \EEE,
 \eeq
%
and we have succeeded in deriving the second fundamental equation
of fluid mechanics.



\subsection{Physical parameters of water}

\begin{table}[h!]
%
\caption[Physical parameters of water]{\tablab{TabMiscWater}
List of various physical constants relating to water at 20$^\circ$C. Data reprinted from \emph{CRC Handbook of Chemistry and
Physics}.}
%
 \centerline{\begin{tabular}{lr@{}l@{}l} \hline
 density &   1.0 & $\times 10^3$&\; $\SIkg\:\SIm^{-3}$ \rule[0mm]{0mm}{5mm}\\
 viscosity &  1.0 & $\times 10^{-3}$&\; $\SIPa\:\SIs$ \\
 surface tension & 72.9 & $\times 10^{-3}$&\; $\SIJ\:\SIm^{-2}$
 \\[1mm] \hline
 \end{tabular}}
%
\end{table}
