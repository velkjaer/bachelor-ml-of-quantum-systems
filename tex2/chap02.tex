%%%%%%%%%%%%%%%%%%%%%%%%%%%%%%%%%%%%%%%%%%%%%%%%%%%%%
%   Chapter 2: Analytical velocity fields           %
%%%%%%%%%%%%%%%%%%%%%%%%%%%%%%%%%%%%%%%%%%%%%%%%%%%%%


\chapter{Analytical Navier--Stokes solutions}
\chaplab{AnalytNSsol}
 \index{Navier--Stokes equation!analytical solutions}
 \index{Navier--Stokes equation!no-slip boundary condition}
 \index{no-slip boundary condition}

The Navier--Stokes equation is notoriously difficult to solve
analytically because it is a non-linear differential equation.
Analytical solutions can however be found in a few, but very
important cases. Some of these solutions are the topic for this
chapter. In particular, we shall solve a number of steady-state
problems, among them Poiseuille flow problems, \ie, pressure
induced steady-state fluid flow in infinitely long,
translation-invariant channels. It is important to study such
idealized flows, since they provide us with basic understanding of
the behavior of liquids flowing in the microchannels of
lab-on-a-chip systems.

Before analyzing the Poiseuille problem we treat three even
simpler flow problems: fluids in mechanical equilibrium, the
gravity-driven motion of a thin liquid film on an inclined plane,
and the motion of a fluid between two parallel plates driven by
the relative motion of theses plates (Couette flow).

In all cases we shall employ the so-called no-slip boundary
condition for the velocity field at the part $\pp \Omega$ of the
boundary that is a solid wall,
%
 \beq{noslip}
 \vvv(\rrr) = \boldsymbol{0},
 \quad \textrm{for}\;\: \rrr \in \pp \Omega \; \;
  \textrm{(no-slip).}
 \eeq
%
The microscopic origin of this condition is the assumption of
complete momentum relaxation between the molecules of the wall,
which are at rest, and the outermost molecules of the fluid that
collide with the wall. The momentum is relaxed on a length scale
of the order the molecular mean free path in the fluid, which for
liquids and high density fluids means one inter-molecular distance
($\simeq0.3$~nm). Only for rarified gases or narrow channels,
where the mean free path of the gas molecules is comparable with
the channel dimensions, is it necessary to abandon the no-slip
boundary condition.


\section{Fluids in mechanical equilibrium} \seclab{FluidsInEqui}
\index{fluids in mechanical equilibrium}
\index{mechanical equilibrium}

A fluid in mechanical equilibrium must be at rest relative to the
walls of the vessel containing it, because otherwise it would
continuously loose kinetic energy by heat conversion due to
internal friction originating from viscous forces inside the
fluid. The velocity field is therefore trivially zero everywhere,
a special case of steady-state\index{steady-state!definition}
defined by $\pp_t\vvv \equiv \bm{0}$. If we let gravity, described
by the gravitational acceleration $\ggn = -g \een_z$ in the
negative $z$ direction, be the only external force, the
Navier--Stokes equation reduces to\index{Navier--Stokes
equation!mechanical equilibrium}
%
 \bsubal
 \eqlab{EquiVzero}
 \vvv(\rrr) &= \boldsymbol{0},\\
 \eqlab{EquiNS}
 \boldsymbol{0} &= -\nablabf p - \rho\:g \een_z.
 \esubal
%
For an imcompressible fluid, say water, \eqref{EquiNS} is easily
integrated to give\index{hydrostatic pressure!incompressible
liquid}
%
 \beq{EquiP}
 p = p^{{}}_0 - \rho\:g z,
 \eeq
%
where $p^{{}}_0$ is the pressure at the arbitrarily defined
zero-level $z=0$. This relation points to an easy way of
generating pressure differences in liquids: the pressure at the
bottom of a liquid column of height $H$ is $\rho\:g H$ higher than
the pressure at height $H$. Liquids with different densities, such
as mercury with $\rho^{{}}_\textrm{Hg} =
1.36\times10^4$~kg$\:$m$^{-3}$ and water with
$\rho^{{}}_\textrm{H$_2$O} = 1.00 \times 10^3$~kg$\:$m$^{-3}$ can
be used to generate different pressures for given heights. The use
of this technique in lab-on-a-chip systems is illustrated in
\exref{LiquidPress}.



Consider a compressible fluid, say, an ideal gas under isothermal
conditions for which
%
 \beq{rhoIdealGas}
 \rho = \frac{\rho^{{}}_0}{p^{{}}_0}\:p,
 \eeq
%
where $\rho^{{}}_0$ and $p^{{}}_0$ is the density and pressure,
respectively, for one particular state of the gas. With this
equation of state \eqsref{EquiVzero}{EquiNS} are changed into
%
 \bsubal
 \eqlab{GasVzero}
 \vvv(\rrr) &= \boldsymbol{0},\\
 \eqlab{GasNS}
 \boldsymbol{0} &= -\nablabf p -
 \frac{\rho^{{}}_0}{p^{{}}_0}\:p g \een_z.
 \esubal
%
Integration of \eqref{GasNS} yields\index{hydrostatic
pressure!compressible liquid}
%
 \beq{GasP}
 p(z) = p^{{}}_0\:
 \exp\Big(-\frac{1}{p^{{}}_0}\: \rho^{{}}_0 g z \Big).
 \eeq
%
Inserting the parameter values for air at the surface of the Earth
(hardly a microfluidic system), the thickness of the atmosphere is
readily estimated to be of the order 10~km; see
\exref{atmosphere}.\index{atmosphere, thickness of}


\section{Liquid film flow on an inclined plane}
\seclab{InclinedPlane}
\index{Navier--Stokes equation!film flow, inclined plane}
\index{inclined plane!Navier--Stokes equation}

The first example of a non-trivial velocity field is that of a
liquid film flowing down along an infinitely long and infinitely
wide inclined plane. Consider the geometry defined in
\figref{InclinedPlane}. The component $g^{{}}_z$ of the
gravitational acceleration normal to the inclined plane is
balanced by the normal forces. The component $g^{{}}_x$ parallel
to the plane accelerates the film down along the inclined plane
until the velocity of the film is so large that the associated
viscous friction forces in the film compensates $g^{{}}_x$. When
this happens the motion of the film has reached steady-state, a
situation we analyze in the following.

\begin{figure}
\centerline{
  \includegraphics[]{figs/chap02/InclinedPlane.eps}}
\caption[Liquid film flow on an inclined
plane]{\figlab{InclinedPlane} A liquid film (light gray) of
uniform thickness $h$ flowing down along an inclined plane (dark
gray). The plane has the inclination angle $\alpha$ and is assumed
to be infinitely long and infinitely wide. The $x$ and $z$ axis is
chosen parallel and normal to the plane, respectively. The
gravitational acceleration is thus given by $\ggn = g \sin\alpha\:
\een_x - g \cos\alpha\: \een_z$. In steady-state the resulting
velocity profile of the liquid film is parabolic as shown.}
\end{figure}

The translation invariance of the setup along the $x$ and $y$
direction dictates that the velocity field can only depend on $z$.
Moreover, since the driving force points along the $x$ direction
only the $x$ component of the velocity field is non-zero. Finally,
no pressure gradients play any role in this free-flow problem, so
the steady-state Navier--Stokes equation (\ie, $\pp_t \vvv = 0$)
becomes
%
 \bsubal
 \eqlab{PlaneV}
 \vvv(\rrr) &= v^{{}}_x(z)\: \een_x,\\
 \eqlab{PlaneNS}
 \rho (\vvv\scap\nablabf)\vvv &=
 \eta\:\ppsqr_z \vvv + \rho\:g \sin\alpha\: \een_x.
 \esubal
%
The special symmetry \eqref{PlaneV} of the velocity field implies
an enormous simplification of the flow problem. Straightforward
differentiation shows namely that the non-linear term in the
Navier--Stokes equation (\ref{eq:PlaneNS}) vanishes,
%
 \beq{NonLinVanish}
 (\vvv\scap\nablabf)\vvv =
 v^{{}}_x(z)\pp_x\big[v^{{}}_x(z)\big] = 0.
 \eeq
%
We thus only  need to solve a linear second-order ordinary
differential equation. This demands two boundary conditions, which
are given by demanding no-slip of $\vvv$ at the plane $z=0$ and no
viscous stress on the free surface, \ie, $\sigma'^{{}}_{xz}$ from
\eqref{ViscousStressTensorInc} is zero at $z=h$. We arrive at
%
 \bsubal
 \eqlab{PlaneNSx}
 \eta\:\ppsqr_z v^{{}}_x(z) &= -\rho\:g \sin\alpha,\\
 v^{{}}_x(0) &= 0, \quad \textrm{(no-slip)}\\
 \eta\:\pp_z v^{{}}_x(h) &= 0, \quad \textrm{(no stress).}
 \esubal
%
The solution is seen to be the well-known half-parabola
%
 \beq{PlaneVparab}
 v^{{}}_x(z) = \sin\alpha\: \frac{\rho\:g}{2\eta}\: z (2h - z).
 \eeq
%
As studied in \exref{PlaneFilm} a typical speed for a 100~$\SImum$
thick film of water is 5~cm/s.




\section{Couette flow}
\seclab{CouetteFlow}
\index{Navier--Stokes equation!Couette flow}
\index{parallel plates, Couette flow}

Couette flow is a flow generated in a liquid by moving one or more
of the walls of the vessel containing the fluid relative to the
other walls. An important and very useful example is the Couette
flow set up in a fluid held in the space between two concentric
cylinders rotating axisymmetrically relative to each other. This
setup is used extensively in rheology\footnote{Rheology is the
study of deformation and flow of matter} because it is possible to
determine the viscosity $\eta$ of the fluid very accurately by
measuring the torque necessary to sustain a given constant speed
of relative rotation.

\begin{figure}
\centerline{
  \includegraphics[]{figs/chap02/CouetteFlow.eps}}
\caption[Couette flow]{\figlab{CouetteFlow} An example of Couette
flow. A fluid is occupying the space of height $h$ between two
horizontally placed, parallel, infinite planar plates. The top
plate is moved with the constant speed $v^{{}}_0$ relative to the
bottom plate. The no-slip boundary condition at both plates forces
the liquid into motion, resulting in the linear velocity profile
shown.}
\end{figure}


Here we study the simpler case of planar Couette flow as
illustrated in \figref{CouetteFlow}. A liquid is placed between
two infinite planar plates. The plates are oriented horizontally
in the $xy$ plane perpendicular to the gravitational acceleration
$\ggn$. The bottom plate at $z=0$ is kept fixed in the laboratory,
while the top plate at $z=h$ is moved in the $x$ direction with
the constant speed $v^{{}}_0$.

As in the previous example there is translation invariance of the
setup along the $x$ and $y$ direction implies that the velocity
field can only depend on $z$. Moreover, since the driving force
points along the $x$ direction, only the $x$ component of the
velocity field is non-zero. Finally, neither body forces nor
pressure forces play a role since both gravity and the only
non-zero pressure gradient (the $z$ component due to hydrostatic
pressure) are compensated by the reaction forces of the bottom
plate. As in the previous example the symmetry again implies
$(\vvv\scap\nablabf)\vvv = 0$, and the steady-state Navier--Stokes
equation then reads
%
 \bsubal
 \eqlab{CouetteV}
 \vvv(\rrr) &= v^{{}}_x(z)\: \een_x,\\
 \eqlab{CouetteNS}
 \eta\:\ppsqr_z \vvv &= \bm{0}.
 \esubal
%
The boundary conditions on $\vvv$ is no-slip at the top and bottom
plane plane $z=0$ and $z=h$, respectively, so we arrive at the
following second-order ordinary differential equation with two
boundary conditions:
%
 \bsub
 \begin{alignat}{2}
 \eqlab{CouetteNSx}
 \eta\:\ppsqr_z v^{{}}_x(z) &= 0, &&\\
 v^{{}}_x(0) &= 0, & & \textrm{(no-slip)}\\
 v^{{}}_x(h) &= v^{{}}_0, \;\; && \textrm{(no-slip).}
 \end{alignat}
 \esub
%
The solution is seen to be the well-known linear profile
%
 \beq{CouetteLin}
 v^{{}}_x(z) = v^{{}}_0\: \frac{z}{h}.
 \eeq
%
Assuming this expression to be valid for large, but finite, plates
with area $\mathcal{A}$ we can, by use of the viscous stress
tensor $\sigma'$, determine the horizontal external force $\FFF =
F^{{}}_x \een_x$ necessary to apply to the top plate to pull it
along with fixed speed $v^{{}}_0$,
%
 \beq{CouetteForce}
 F^{{}}_x = \sigma'^{{}}_{xz}\: \mathcal{A} =
 \eta\: \frac{v^{{}}_0\mathcal{A}}{h}.
 \eeq
%
This expression allows for a simple experimental determination of
the viscosity $\eta$.


\section{Poiseuille flow}
\seclab{PoiseFlow}
\index{Navier--Stokes equation!Poiseuille flow}
\index{Pressure driven flow, Poiseuille flow}
\index{velocity field!Poiseuille flow}

We now turn to the final class of analytical solutions to the
Navier--Stokes equation: the pressure-driven, steady-state flows
in channels, also known as Poiseuille flows or Hagen-Poiseuille
flows.  This class is of major importance for the basic
understanding of liquid handling in lab-on-a-chip systems



In a Poiseuille flow the fluid is driven through a long, straight,
and rigid channel by imposing a pressure difference between the
two ends of the channel. Originally, Hagen and Poiseuille studied
channels with circular cross-sections, as such channels are
straightforward to produce. However, especially in microfluidics,
one frequently encounters other shapes. One example is the
Gaussian-like profile that results from producing microchannels by
laser ablation in the surface of a piece of the polymer PMMA. The
heat from the laser beam cracks the PMMA into MMA, which by
evaporation leaves the substrate. A whole network of microchannels
can then be created by sweeping  the laser beam across the
substrate in a well-defined pattern. The channels are sealed by
placing and bonding a polymer lid on top of the structure.


\subsection{Arbitrary cross-sectional shape}
\seclab{PoiseGeneral}
\index{arbitrary cross-section, Poiseuille flow}
\index{Poiseuille flow!arbitrary cross-section}

We first study the steady-state Poiseuille flow problem with an
arbitrary cross-sectional shape as illustrated in
\figref{PoiseGeneral}. Although not analytically solvable, this
example nevertheless provide us with the structural form of the
solution for the velocity field.

\begin{figure}
\centerline{
  \includegraphics[]{figs/chap02/PoiseGeneral.eps}}
\caption[Poiseuille flow: arbitrary
cross-section]{\figlab{PoiseGeneral} The Poiseuille flow problem
in a channel, which is translation invariant in the $x$ direction,
and which has an arbitrarily shaped cross-section $\mathcal{C}$ in
the $yz$ plane. The boundary of $\mathcal{C}$ is denoted $\pp
\mathcal{C}$. The pressure at the left end, $x = 0$, is an amount
$\Delta p$ higher than at the right end, $x = L$.}
\end{figure}


The channel is parallel to the $x$ axis, and it is assumed to be
translation invariant in that direction. The constant
cross-section in the $yz$ plane is denoted $\mathcal{C}$ with
boundary $\pp\mathcal{C}$, respectively. A constant pressure
difference $\Delta p$ is maintained over a segment of length $L$
of the channel, \ie, $p(0) = p^{{}}_0+\Delta p$ and $p(L) =
p^{{}}_0$. The gravitational force is balanced by a hydrostatic
pressure gradient in the vertical direction. These two forces are
therefore left out of the treatment. The translation invariance of
the channel in the $x$ direction combined with the vanishing of
forces in the $yz$ plane implies that the velocity field cannot
depend on $x$, while only its $x$ component can be non-zero. This
implies once more that $(\vvv\scap\nablabf)\vvv = 0$ and the
steady-state Navier--Stokes equation thus becomes
%
 \bsubal
 \eqlab{PoiseGenV}
 \vvv(\rrr) &= v^{{}}_x(y,z)\: \een_x,\\
 \eqlab{PoiseGenNS}
 \bm{0} &=
 \eta \nablabf^2 \big[ v^{{}}_x(y,z)\: \een_x \big]
 -\nablabf p.
 \esubal
%
Since the $y$ and $z$ components of the velocity field are zero,
it follows that $\pp_yp = 0$ and $\pp_zp = 0$, and consequently
that the pressure field only depends on $x$, $p(\rrr) = p(x)$.
Using this result the $x$ component of the Navier--Stokes
equation~(\ref{eq:PoiseGenNS}) becomes
%
 \beq{PoiseGenNSx0}
 \eta \big[\ppsqr_y+\ppsqr_z\big] v^{{}}_x(y,z) =
 \pp_x p(x).
 \eeq
%
Here it is seen that the left-hand side is a function of $y$ and
$z$ while the right-hand side is a function of $x$. The only
possible solution is thus that the two sides of the Navier--Stokes
equation equal the same constant. However, a constant pressure
gradient $\pp_x p(x)$ implies that the pressure must be a linear
function of $x$, and using the boundary conditions for the
pressure we obtain
%
 \beq{PoiseGenPress}
 p(\rrr) = \frac{\Delta p}{L}\: (L-x) + p^{{}}_0.
 \eeq
%
With this we finally arrive at the second-order partial
differential equation that $v^{{}}_x(y,z)$ must fulfil in the
domain $\mathcal{C}$ given the usual no-slip boundary conditions
at the solid walls of the channel described by $\pp\mathcal{C}$,
%
 \bsub
 \begin{alignat}{2}
 \eqlab{PoiseGenNSx}
 \big[\ppsqr_y+\ppsqr_z\big] v^{{}}_x(y,z)
 & = - \: \frac{\Delta p}{\eta L},\;\;& &
 \textrm{for} \; (y,z)\in \mathcal{C}\\
 \eqlab{PoiseBC}
 v^{{}}_x(y,z) &= 0, & & \textrm{for} \; (y,z)\in\pp \mathcal{C}.
 \end{alignat}
 \esub
%

Once the velocity field is determined it is possible to calculate
the so-called flow rate $Q$, which is defined as the fluid volume
discharged by the channel per unit time. For compressible fluids
its becomes important to distinguish between the flow rate $Q$ and
the mass flow rate $Q^{{}}_\textrm{mass}$ defined as the
discharged mass per unit time. In the case of the geometry of
\figref{PoiseGeneral} we have\index{flow rate, definition}
\index{mass flow rate}
%
 \bsubal
 \eqlab{Qdef}
 Q & \equiv \int_\mathcal{C}dy\:dz\:
 v^{{}}_x(y,z),\\[2mm]
 \eqlab{Qmassdef}
 Q^{{}}_\textrm{mass} & \equiv \int_\mathcal{C}dy\:dz\:
 \rho\: v^{{}}_x(y,z).
 \esubal
%
This is how far we can get theoretically without specifying the
actual shape of the channel.


\subsection{Elliptic cross-section}
\seclab{Elliptic}
\index{Poiseuille flow!elliptic cross-section}

Our first explicit example is the elliptic cross-section. We let
the center of the ellipse be at $(y,z) = (0,0)$. The major axis of
length $a$ and the minor axis of length $b$ are parallel to the
$y$ axis and $z$ axis, respectively, as shown in
\figref{EllipseCircTri}(a). The boundary $\pp \mathcal{C}$ of the
the ellipse is given by the expression
%
 \beq{EllipseDC}
 \pp \mathcal{C}:\quad 1 - \frac{y^2}{a^2} - \frac{z^2}{b^2} = 0.
 \eeq
%
If we therefore as a trial solution choose
%
 \beq{ellipseVtrial}
 v^{{}}_x(y,z) = v^{{}}_0\:
 \bigg(1 - \frac{y^2}{a^2} - \frac{z^2}{b^2} \bigg),
 \eeq
%
we are guarantied that $v^{{}}_x(y,z)$ satisfies the no-slip
boundary condition \eqref{PoiseBC}. Insertion of the trial
solution into the left-hand side of the Navier--Stokes equation
(\ref{eq:PoiseGenNSx}) yields
%
 \beq{EllipseNStrial}
 \big[\ppsqr_y+\ppsqr_z\big] v^{{}}_x(y,z) =
 -2v^{{}}_0\: \bigg(\frac{1}{a^2} + \frac{1}{b^2} \bigg).
 \eeq
%
Thus the Navier-Stokes equation will be satisfied by choosing the
constant $v^{{}}_0$ as
%
 \beq{EllipseV0}
 v^{{}}_0 = \frac{\Delta p}{2\eta L}\: \frac{a^2b^2}{a^2+b^2}\: .
 \eeq
%
To calculate the flow rate $Q$ for the elliptic channel we need to
evaluate a 2d integral in an elliptically shaped integration
region. This we handle by the following coordinate transformation.
Let $(\rho,\phi)$ be the polar coordinates of the unit disk, \ie,
the radial and azimuthal coordinates obey $0\leq\rho\leq 1$ and
$0\leq\phi\leq 2\pi$, respectively. Our physical coordinates
$(y,z)$ and the velocity field $v^{{}}_x$ can then be expressed as
functions of $(\rho,\phi)$:\index{polar coordinates}
%
 \bal
 \eqlab{EllipseYtrans}
 y(\rho,\phi) &= a \rho\: \cos\phi,\\
 \eqlab{EllipseZtrans}
 z(\rho,\phi) &= b \rho\: \sin\phi,\\
  \eqlab{EllipseVtrans}
 v^{{}}_x(\rho,\phi) &= v^{{}}_0 \big(1- \rho^2\big).
 \eal
%
The advantage is that now the boundary $\pp \mathcal{C}$ can be
expressed in terms of just one coordinate instead of two,
%
 \beq{EllipseDCtrans}
 \pp \mathcal{C}:\quad \rho = 1.
 \eeq
%
The $(y,z)$ surface integral in \eqref{Qdef} is transformed into
$(\rho,\phi)$ coordinates by use of the Jacobian determinant
$|\pp_{(\rho,\phi)} (y,z)|$,\index{Jacobian determinant}
%
 \bal
 \nn
 \int_\mathcal{C} dy\:dz
 &= \int_\mathcal{C} d\rho\:d\phi\:
 \bigg|\frac{\pp(y,z)}{\pp(\rho,\phi)}\bigg|
 = \int_\mathcal{C} d\rho\:d\phi\: \bigg|\begin{array}{cc}
 \pp_\rho y & \pp_\rho z\\ \pp_\phi y & \pp_\phi z
 \end{array} \bigg| \\[2mm]
 \eqlab{EllipseJacobian}
 &= \int_0^1 d\rho \int_0^{2\pi} d\phi\:
 \bigg|\begin{array}{rr}
 +a\cos\phi & +b\sin\phi\\ -a\rho\sin\phi & +b\rho\cos\phi
 \end{array} \bigg|
 = a b \int_0^{2\pi} d\phi \int_0^1 d\rho\: \rho.
 \eal
%
The flow rate $Q$ for the elliptic channel is now easily
calculated as
%
 \beq{EllipseQ}
 Q = \int_\mathcal{C} dy\:dz\: v^{{}}_x(y,z) =
 a b \int_0^{2\pi} d\phi \int_0^1 d\rho\: \rho\:
 v_x(\rho,\phi) =
 \frac{\pi}{4}\: \frac{1}{\eta L}\:
 \frac{a^3 b^3}{a^2+b^2} \: \Delta p.
 \eeq
%


\subsection{Circular cross-section}
\seclab{Circular}
\index{Poiseuille flow!circular cross-section}

Since the circle \figref{EllipseCircTri}(b) is just the special
case $a=b$ of the ellipse, we can immediately write down the
result for the velocity field and flow rate for the Poiseuille
flow problem in a circular channel. From
Eqs.~(\ref{eq:ellipseVtrial}), (\ref{eq:EllipseV0}),
and~(\ref{eq:EllipseQ}) using $a=b$ it follows that
%
 \bsubal
 \eqlab{CircV}
 v^{{}}_x(y,z) &= \frac{\Delta p}{4\eta L}\: \Big(a^2-y^2-z^2),\\
 \eqlab{CircQ}
 Q &= \frac{\pi a^4}{8\eta L}\: \Delta p.
 \esubal
%
However, the same result can also be obtained by direct
calculation using cylindrical coordinates $(x,r,\phi)$ thereby
avoiding the trial solution \eqref{ellipseVtrial}. For cylindrical
coordinates with the $x$ axis as the cylinder axis we have
\index{Laplace operator!cylindrical coordinates}
%
\begin{figure}
\centerline{
  \includegraphics[]{figs/chap02/EllipseCircTri.eps}}
\caption[Poiseuille flow: ellipse, circle, triangle]
{\figlab{EllipseCircTri} The definition of three specific
cross-sectional shapes for the Poiseuille flow problem. (a) The
ellipse with major axis $a$ and minor axis $b$, (b) the circle
with radius $a$, and (c) the equilateral triangle with side length
$a$.}
\end{figure}
%
 \bsubal
 \eqlab{CartCylCoord}
 (x,\: y,\: z) &= (x,\: r\cos\phi,\: r\sin\phi),\\
 \een_x    &= \een_x,\\
 \een_r    &= +\cos\phi\:\een_y + \sin\phi\: \een_z,\\
 \een_\phi &= -\sin\phi\:\een_y + \cos\phi\: \een_z,\\
 \nablabf^2 &=
   \ppsqr_x + \ppsqr_r + \frac{1}{r}\pp_r + \frac{1}{r^2}\pp_\phi.
 \esubal
%
The symmetry considerations reduces the velocity field to $\vvv =
v^{{}}_x(r)\: \een_x$, so that the Navier-Stokes
equation~(\ref{eq:PoiseGenNSx}) becomes an ordinary differential
equation of second order,
%
 \beq{CircNScyl}
 \Big[\ppsqr_r + \frac{1}{r}\:\pp_r\Big] v^{{}}_x(r) =
 -\frac{\Delta p}{\eta L}.
 \eeq
%
The solutions to this inhomogeneous equation\index{inhomogeneous
differential equation} is the sum of a general solution to the
homogeneous equation\index{homogeneous differential equation},
$v''^{{}}_x + v'^{{}}_x/r = 0$, and one particular solution to the
inhomogeneous equation. It is easy to see that the general
homogeneous solution has the linear form $v^{{}}_x(r) = A + B\ln
r$, while a particular inhomogeneous solution is $v^{{}}_x(r) =
-(\Delta p/4\eta L)\: r^2$. Given the boundary conditions
$v^{{}}_x(a) = 0$ and $v'^{{}}_x(0) = 0$ we arrive at
%
 \bsubal
 \eqlab{CircVcyl}
 v^{{}}_x(r,\phi) &= \frac{\Delta p}{4\eta L} \: \big(a^2 - r^2\big)\\
 \eqlab{CircQcyl}
 Q &= \int_0^{2\pi}d\phi \int_0^a dr\: r\:
 \frac{\Delta p}{4\eta L} \: \big(a^2 - r^2)
 = \frac{\pi}{8}\: \frac{a^4}{\eta L}\: \Delta p.
 \esubal


\subsection{Equilateral triangular cross-section}
\seclab{Triangular}
\index{Poiseuille flow!triangular cross-section}

There exists no analytical solution to the Poiseuille flow problem
with a general triangular cross-section. In fact, it is only for
the equilateral triangle defined in \figref{EllipseCircTri}(c)
that an analytical result is known.

The domain $\mathcal{C}$ in the $yz$ plane of the equilateral
triangular channel cross-section can be thought of as the union of
the three half-planes $(\sqrt{3}/2)a \geq z$, $z \geq \sqrt{3} y$,
and $z \geq -\sqrt{3} y$. Inspired by our success with the trial
solution of the elliptic channel, we now form a trial solution by
multiplying together the expression for the three straight lines
defining the boundaries of the equilateral triangle,
%
 \beq{TriVtrial}
 v^{{}}_x(y,z) = \frac{v^{{}}_0}{a^3}\:
 \Big(\frac{\sqrt{3}}{2}a - z\Big)
 \Big(z-\sqrt{3}y\Big)
 \Big(z+\sqrt{3}y\Big) =
 \frac{v^{{}}_0}{a^3}\:
 \Big(\frac{\sqrt{3}}{2}a - z\Big)
 \Big(z^2-3y^2\Big).
 \eeq
%
Per construction this trial solution satisfies the no-slip
boundary condition on $\pp \mathcal{C}$. Luckily, it turns out
that the Laplacian acting on the trial solution yields a constant,
%
 \beq{TriNStrial}
 \big[\ppsqr_y+\ppsqr_z\big] v^{{}}_x(y,z) =
 -2\sqrt{3}\: \frac{v^{{}}_0}{a^2}.
 \eeq
%
Thus the Navier-Stokes equation will be satisfied by choosing the
constant $v^{{}}_0$ as
%
 \beq{TriV0}
 v^{{}}_0 = \frac{1}{2\sqrt{3}}\: \frac{\Delta p}{\eta L}\: a^2.
 \eeq
%
The flow rate $Q$ is most easily found by first integrating over
$y$ and then over $z$,
%
 \bal
 \nn
 Q &= 2\int_0^{\frac{\sqrt{3}}{2}a} dz
      \int_0^{\frac{1}{\sqrt{3}}\:z} dy\: v^{{}}_x(y,z)
   = \frac{4v^{{}}_0}{3\sqrt{3}\:a^3}
   \int_0^{\frac{\sqrt{3}}{2}a} dz
   \bigg(\frac{\sqrt{3}}{2}\:a - z\bigg)\: z^3\\
 \eqlab{TriQ}
 &= \frac{3}{160}\: v^{{}}_0\: a^2 =
 \frac{\sqrt{3}}{320}\: \frac{a^4}{\eta L}\: \Delta p.
 \eal
%


\subsection{Rectangular cross-section}
\seclab{Rectangular}
\index{Poiseuille flow!rectangular cross-section}

For lab-on-a-chip systems many fabrication methods leads to
microchannels having a rectangular cross-section. One example is
the microreactor shown in panel (a) and (b) of \figref{Rectangle}.
This device is made in the polymer SU-8 by hot embossing, \ie, the
SU-8 is heated up slightly above its glass transition temperature,
where it gets soft, and then a hard stamp containing the negative
of the desired pattern is pressed into the polymer. The stamp is
removed and later a polymer lid is placed on top of the structure
and bonded to make a leakage-free channel.

It is perhaps a surprising fact that no analytical solution is
known to the Poiseuille flow problem with a rectangular
cross-section. In spite of the high symmetry of the boundary the
best we can do analytically is to find a Fourier sum representing
the solution.

\begin{figure}
\centerline{
  \includegraphics[width=\textwidth]{figs/chap02/Rectangle.eps}}
\caption[Poiseuille flow: rectangle]{\figlab{Rectangle} (a) A
top-view picture of a micro-reactor with nine inlet microchannels
made by hot embossing in the polymer SU-8 before bonding on a
polymer lid. (b) A zoom-in on one of the inlet channels having a
near perfect rectangular shape of height $h=50~\SImum$ and width
$w = 100~\SImum$ . Courtesy the group of Geschke at DTU Nanotech. (c) The definition of the rectangular channel cross-section of height $h$
and width $w$, which is analyzed in the text.\index{lab-on-a-chip
systems!microreactor on PMMA chip}}
\end{figure}

In the following we always take the width to be larger than the
height, $w>h$. By rotation this situation can always be realized.
The Navier--Stokes equation and associated boundary conditions are
%
 \bsub
 \begin{alignat}{2}
 \eqlab{RectNSx}
 \big[\ppsqr_y+\ppsqr_z\big] v^{{}}_x(y,z)
 & = - \: \frac{\Delta p}{\eta L},& & \quad
 \textrm{for} \; \; -\frac{1}{2}w<y<\frac{1}{2}w, \;\; 0 < z < h,\\
 \eqlab{RectBC}
 v^{{}}_x(y,z) &= 0, & & \quad \textrm{for} \; \;
 y = \pm \frac{1}{2}w, \; z=0, \; z=h.
 \end{alignat}
 \esub
%
We begin by expanding all functions in the problem as Fourier
series along the short vertical $z$ direction. To ensure the
fulfilment of the boundary condition $v_x(y,\pm h) = 0$ we use
only terms proportional to $\sin( n\pi z/h)$, where $n$ is a
positive integer. A Fourier expansion of the constant on the
right-hand side in \eqref{RectNSx} yields,\index{Fourier series
for velocity field}\index{velocity field!Fourier expansion}
%
 \beq{FourierP}
 - \: \frac{\Delta p}{\eta L} = - \: \frac{\Delta p}{\eta L}\:
 \frac{4}{\pi} \sum_ {n,\textrm{odd}}^\infty \frac{1}{n}\:
 \sin\Big( n\pi \frac{z}{h}\Big),
 \eeq
%
a series containing only odd integers $n$. The coefficients
$f^{{}}_n(y)$ of the Fourier expansion in the $z$ coordinate of
the velocity are constants in $z$, but functions in $y$:
%
 \beq{FourierV}
 v^{{}}_x(y,z) \equiv
 \sum_ {n=1}^\infty f^{{}}_n(y)
 \sin\Big( n\pi \frac{z}{h}\Big).
 \eeq
%
Inserting this series in the left-hand side of \eqref{RectNSx}
leads to
%
 \beq{FourierLapV}
 \big[\ppsqr_y+\ppsqr_z\big] v^{{}}_x(y,z)
 =  \sum_ {n=1}^\infty \Big[f''_n(y) -
 \frac{n^2\pi^2}{h^2}\: f^{{}}_n(y) \Big]\:
 \sin\Big( n\pi \frac{z}{h}\Big).
 \eeq
%
A solution to the problem must satisfy that for all values of $n$
the $n$th coefficient in the pressure term \eqref{FourierP} must
equal the $n$th coefficient in the velocity term
\eqref{FourierLapV}. The functions $f^{{}}_n(y)$ are therefore
given by
%
\begin{figure}
\centerline{
  \includegraphics[width=\textwidth]{figs/chap02/RectVcontour.eps}}
\caption[Flow profile in rectangular
channels]{\figlab{RectVcontour} (a) Contour lines for the velocity
field $v^{{}}_x(y,z)$ for the Poiseuille flow problem in a
rectangular channel. The contour lines are shown in steps of 10\%
of the maximal value $v^{{}}_x(0,h/2)$. (b) A plot of
$v^{{}}_x(y,h/2)$ along the long center-line parallel to $\een_y$.
(c) A plot of $v^{{}}_x(0,z)$ along the short center-line parallel
to $\een_z$.}
\end{figure}
%
 \bsub
 \begin{alignat}{2}
 \eqlab{FourierNeven}
 f^{{}}_n(y) &= 0, &&\quad \textrm{for $n$ even},\\
 \eqlab{FourierNodd}
 f''_n(y) -  \frac{n^2\pi^2}{h^2}\: f^{{}}_n(y) &=
 - \: \frac{\Delta p}{\eta L}\: \frac{4}{\pi}\: \frac{1}{n},
 & &\quad  \textrm{for $n$ odd.}
 \end{alignat}
 \esub
%
To determine $f^{{}}_n(y)$, for $n$ being odd, we need to solve
the inhomogeneous second order differential equation
(\ref{eq:FourierNodd}).\index{inhomogeneous differential equation}
\index{homogeneous differential equation} A general solution can
be written as
%
 \beq{fnGeneral}
 f^{{}}_n(y) = f^{{\textrm{inhom}}}_n(y) +
 f^{{\textrm{homog}}}_n(y),
 \eeq
%
where $f^{{\textrm{inhom}}}_n(y)$ is a particular solution to the
inhomogeneous equation and $f^{{\textrm{homog}}}_n(y)$ a general
solution to the homogeneous equation (where the right-hand side is
put equal to zero). It is easy to find one particular solution to
\eqref{FourierNodd}. One can simply insert the trial function
$f^{{\textrm{inhom}}}_n(y) = \textrm{const}$ and solve the
resulting algebraic equation,
%
 \beq{fnInhom}
 f^{{\textrm{inhom}}}_n(y) =
 \frac{4h^2 \Delta p }{\pi^3 \eta L}\: \frac{1}{n^3},
 \quad  \textrm{for $n$ odd.}
 \eeq
%
The general solution to the homogeneous equation, $ f''_n(y) -
(n^2\pi^2/h^2)\: f^{{}}_n(y) = 0$ is the linear combination
%
 \beq{fnHomog}
 f^{{\textrm{homog}}}_n(y) =
 A\cosh\Big(\frac{n\pi}{h}\: y\Big) +
 B\sinh\Big(\frac{n\pi}{h}\: y\Big).
 \eeq
%
The solution $f^{{}}_n(y)$ that satisfies the no-slip boundary
conditions $f^{{}}_n\big(\pm\frac{1}{2}w\big) = 0$ is
%
 \beq{fnSolution}
 f^{{}}_n(y) =
 \frac{4h^2 \Delta p}{\pi^3 \eta L}\: \frac{1}{n^3}\:
 \Bigg[1 -
 \frac{\cosh\big(n\pi\frac{y}{h}
 \big)}{\cosh\big(n\pi\frac{w}{2h}\big)}\Bigg] ,
 \quad  \textrm{for $n$ odd,}
 \eeq
%
which leads to the velocity field for the Poiseuille flow in a
rectangular channel,
%
 \beq{RectV}
 v^{{}}_x(y,z) =
 \frac{4h^2 \Delta p}{\pi^3 \eta L}\:
 \sum_{n,\textrm{odd}}^\infty \frac{1}{n^3}\: \Bigg[1 -
 \frac{\cosh\big(n\pi\frac{y}{h}
 \big)}{\cosh\big(n\pi\frac{w}{2h}\big)}\Bigg]
 \sin\Big(n\pi\frac{z}{h}\Big).
 \eeq
%
In \figref{RectVcontour} are shown some plots of the contours of
the velocity field and of the velocity field along the symmetry
axes.

The flow rate Q is found by integration as follows,
%
 \bal
 Q &= 2\int_0^{\frac{1}{2}w}dy\: \int_0^h dz\:
 v^{{}}_x(y,z) \nn\\
 &= \frac{4h^2 \Delta p}{\pi^3 \eta L}\:
 \sum_{n,\textrm{odd}}^\infty
 \frac{1}{n^3}\: \frac{2h}{n\pi} \: \Bigg[w -
 \frac{2h}{n\pi}\tanh\Big(n\pi\frac{w}{2h}\Big)\Bigg] \nn\\
 &= \frac{8h^3w \Delta p}{\pi^4 \eta L}\:
 \sum_{n,\textrm{odd}}^\infty  \Bigg[\frac{1}{n^4} -
 \frac{2h}{\pi w}\:\frac{1}{n^5}
 \tanh\Big(n\pi\frac{w}{2h}\Big)\Bigg]\nn\\
 \eqlab{RectQ}
 &= \frac{h^3w \Delta p}{12 \eta L}\:
 \Bigg[1 - \sum_{n,\textrm{odd}}^\infty
 \frac{1}{n^5}\:\frac{192}{\pi^5}\:\frac{h}{w}\:
 \tanh\Big(n\pi\frac{w}{2h}\Big)\Bigg],
 \eal
%
where we have used $\dpst \sum_{n,\textrm{odd}}^\infty
 \frac{1}{n^4} = \frac{\pi^4}{96}$.

Very useful approximate results can be obtained in the limit
$\frac{h}{w}\rightarrow 0$ of a flat and very wide channel, for
which $\frac{h}{w} \tanh\big(n\pi\frac{w}{2h}\big) \rightarrow
\frac{h}{w}\tanh(\infty) = \frac{h}{w}$, and $Q$ becomes
%
 \bal
 Q
 &\approx \frac{h^3w \Delta p}{12 \eta L}\:
 \Bigg[1 - \frac{192}{\pi^5}\:\frac{h}{w}
 \sum_{n,\textrm{odd}}^\infty \frac{1}{n^5}\:
 \Bigg] \nn\\
 &= \frac{h^3w \Delta p}{12 \eta L}\:
 \Big[1 -
 \frac{192}{\pi^5}\:\frac{31}{32}\:\zeta(5)\:\frac{h}{w}
 \Big] \nn\\
 \eqlab{RectQapprox}
 &\approx \frac{h^3w \Delta p}{12 \eta L}\:
 \Big[1 - 0.630\: \frac{h}{w}\Big].
 \eal
%
Here we have on the way used the Riemann zeta
function,\index{Riemann zeta function} $\zeta(x) \equiv
\sum_{n=1}^{\infty} 1/n^x$,
%
 \beq{RiemannZeta}
 \sum_{n,\textrm{odd}}^\infty \frac{1}{n^5}
 = \sum_{n=1}^\infty \frac{1}{n^5} -
 \sum_{n,\textrm{even}}^\infty \frac{1}{n^5}
 = \zeta(5) - \sum_{k=1}^\infty \frac{1}{(2k)^5}
 = \zeta(5) - \frac{1}{32}\:\zeta(5) = \frac{31}{32}\:\zeta(5).
 \eeq
%
The approximative result \eqref{RectQapprox} for $Q$ is
surprisingly good. For the worst case, the square with $h=w$, the
error is just 13\%, while already at aspect ratio a half, $h=w/2$,
the error is down to 0.2\%.

If we neglect the side walls completely, \ie, treat the case of an
infinitely wide channel, the flow rate trough a section of width
$w$ of such a channel is (see also \exref{PoiseInfParallel})
%
 \beq{InfWideQ}
 Q \approx \frac{h^3w \Delta p}{12 \eta L}.
 \eeq
%
This error of this result is off by 23\% for aspect ratio one
third, $h=w/3$, and by 7\% for aspect ration one tenth, $h=w/10$.
So it is better to use our first approximative result for $Q$
\eqref{RectQapprox}, where the effects of the side walls are
included.


\subsection{Infinite parallel-plate channel}
\seclab{InfPlate}
\index{Poiseuille flow!infinite parallel-plate cross-section}

In microfluidics the aspect ratio of a rectangular channel can
often be so large that the channel is well approximated by an
infinite parallel-plate configuration. The geometry is much like
the one shown for the Couette flow in \figref{CouetteFlow}, but
now the both plates are kept fixed and a pressure difference
$\Delta p$ is applied. Due to the symmetry the $y$ coordinate
drops out and we end with the following ordinary differential
equation,
%
 \bsub
 \begin{alignat}{2}
 \eqlab{InfPlateNSx}
 \ppsqr_z v^{{}}_x(z) &= -\frac{\Delta p}{\eta L}, &&\\
 v^{{}}_x(0) &= 0, & & \textrm{(no-slip)}\\
 v^{{}}_x(h) &= 0, & & \textrm{(no-slip).}
 \end{alignat}
 \esub
%
The solution is a simple parabola
%
 \beq{InfPlateVx}
 v^{{}}_x(z) = \frac{\Delta p}{2\eta L}\: (h-z)z,
 \eeq
%
and the flow rate $Q$ through a section of width $w$ is found as
%
 \beq{InfPlateQ}
 Q = \int_0^w dy \int_0^h dz\: \frac{\Delta p}{2\eta L}\: (h-z)z
 = \frac{h^3 w}{12\eta L}\: \Delta p.
 \eeq
%

\section{Shape perturbation in Poiseuille flow problems}
\seclab{ShapePert}
\index{Poiseuille flow!perturbation of circular shape}
\index{shape perturbation of Poiseuille flow}
\index{perturbation theory}

By use of shape perturbation theory it is possible to extend the
analytical results for Poiseuille flow beyond the few cases of
regular geometries that we have treated above. In shape
perturbation theory the starting point is an analytically solvable
case, which then is deformed slightly characterized by some small
perturbation parameter $\epsilon$. As illustrated in
\figref{ShapePert} the unperturbed shape is described by
parametric coordinates $(\eta,\zeta)$ in Cartesian form or
$(\rho,\theta)$ in polar form. The coordinates of the physical
problem we would like to solve are $(y,z)$ in Cartesian form and
$(r,\phi)$ in polar form.


\begin{figure}
\centerline{
  \includegraphics[width=\textwidth]{figs/chap02/ShapePert.eps}}
\caption[Shape perturbation of circular
channel]{\figlab{ShapePert} (a) The geometry of the unperturbed
and analytically solvable cross-section, the unit circle,
described by coordinates $(\eta,\zeta)$ or $(\rho,\theta)$. (b)
The geometry of the perturbed cross-section described by
coordinates $(y,z)$ or $(r,\phi)$ and the perturbation parameter
$\epsilon$. Here $a=1$, $k=5$ and $\epsilon=0.2$.}
\end{figure}

As a concrete example we take the multipolar deformation of the
circle defined by the transformation
%
 \bsub
 \begin{alignat}{2}
 \eqlab{PertTransformPhi}
 \phi &= \theta,&&\quad 0 \leq \theta \leq 2\pi,\\
 \eqlab{PertTransformR}
 r  &= a\:\rho\big[1 + \epsilon \sin(k\theta)\big],&&\quad  0\leq\rho\leq 1,\\
 \eqlab{PertTransformY}
 y(\rho,\theta)  &= a\:\rho\big[1 + \epsilon \sin(k\theta)\big]\cos\theta,&&\\
 \eqlab{PertTransformZ}
 z(\rho,\theta)  &= a\:\rho\big[1 + \epsilon
 \sin(k\theta)\big]\sin\theta,&&
 \end{alignat}
 \esub
%
where $a$ is length scale and $k$ is an integer defining the order
of the multipolar deformation. Note that for $\epsilon=0$ the
shape is unperturbed. The boundary of the perturbed shape is
simply described by fixing the the unperturbed coordinate $\rho=1$
and sweeping in $\theta$
%
 \beq{PertBoundary}
 \pp C: \quad \big(y,z\big) =
 \big(y(1,\theta),\: z(1,\theta) \big).
 \eeq
%
It is therefore desirable to formulate the perturbed Poiseuille
problem using the unperturbed coordinates. To obtain analytical
results it is important to make the appearance of the perturbation
parameter explicit. When performing a perturbation calculation to
order $m$ all terms containing $\epsilon^l$ with $l>m$ are
discarded, while the remaining terms containing the same power of
$\epsilon$ are grouped together, and the equations are solved
power by power.

To carry out the perturbation calculation the velocity field
$v^{{}}_x(y,z)$ is written as
\index{Taylor expansion!shape perturbation}
%
 \beq{PertV}
 v^{{}}_x(y,z) = v^{{}}_x\big(y(\rho,\theta),z(\rho,\theta)\big)
 = v^{{(0)}}_x(\rho,\theta) + \epsilon\:  v^{{(1)}}_x(\rho,\theta)
 + \epsilon^2\:  v^{{(2)}}_x(\rho,\theta) + \cdots
 \eeq
%
Likewise, the Laplacian operator in the Navier--Stokes equation
must be expressed in terms of $\rho$, $\theta$, and $\epsilon$.
The starting point of this transformation is the transformation of
the gradients
%
 \bsubal
 \eqlab{PertGradR}
 \pp_r &= (\pp_r \rho)\:\pp_\rho + (\pp_r \theta)\:\pp_\theta,\\
 \eqlab{PertGradPhi}
 \pp_\phi &= (\pp_\phi \rho)\:\pp_\rho + (\pp_\phi
 \theta)\:\pp_\theta.
 \esubal
%
The derivatives $(\pp_r \rho)$, $(\pp_r \theta)$, $(\pp_\phi
\rho)$, and $(\pp_\phi \theta)$ is obtained from the inverse
transformation of \eqsref{PertTransformR}{PertTransformPhi},
%
 \bsubal
 \eqlab{PertTransformRho}
 \rho(r,\phi)  &= \frac{1}{1 + \epsilon \sin(k\phi)}\: \frac{r}{a}, \\
 \eqlab{PertTransformTheta}
 \theta(r,\phi) &= \phi.
 \esubal
%
The expansion \eqref{PertV} can now be inserted into the
Navier--Stokes equation and by use of the derivatives
\eqsref{PertGradR}{PertGradPhi} we can carry out the perturbation
scheme. The calculation is straightforward but tedious. We shall
here just quote the first order perturbation result for the
velocity field:
%
 \beq{PertVfirst}
 v^{{}}_x(\rho,\theta) = \Big[
 \big(1-\rho^2\big)
 - 2
 \big(\rho^2-\rho^k\big)\sin(k\theta)\:\epsilon
 \Big]\: \frac{a^2 \Delta p}{4\eta L} + \mathcal{O}\big(\epsilon^2\big).
 \eeq
%
This example may appear rather artificial. However, almost any
shape deformation of the circle can by analyzed based on this
example. An arbitrarily shaped boundary can be written as a
Fourier series involving a sum over infinitely many multipole
deformations like the $k$th one studied in this section.
