

\chapter{Fourier transformations I}
\seclab{FourierTrans}
%\setcounter{page}{1}
\index{Fourier transformation!basic theory}

Fourier transformation is useful to employ in the case of homogeneous
systems or to change linear differential equations into linear
algebraic equations. The idea is to resolve the quantity $f(\rrr,t)$
under study on plane wave components,
\beq{PlaneWave}
\ffn_{\kkk,\omega}\: e^{i(\kkk\cdot\rrr - \omega t)},
\eeq
travelling at the speed $v = \omega/|\kkk|$.

\section{Continuous functions in a finite region}
\seclab{FiniteRegion}

Consider a rectangular box in 3D with side lengths $L_x$, $L_y$, $L_z$
and a volume $\vol = L_x L_y L_z$. The central theorem in Fourier
analysis states that any well-behaved function fulfilling the periodic
boundary conditions,
\beq{PeriodicBC}
f(\rrr+L_x\eee_x) = f(\rrr+L_y\eee_y) = f(\rrr+L_z\eee_z) = f(\rrr)
\end{equation}
can be written as a Fourier series
\beq{fk_sum}
f(\rrr) = \frac{1}{\vol}\sum_{\kkk} \ffn_{\kkk}\: e^{i\kkk\cdot\rrr},\;
\left\{ \begin{array}{l}
k_x = \frac{2\pi n_x}{L_x},\; n_x = 0, \pm 1, \pm 2, \ldots\\
{\rm likewise\; for}\;\: y\;\: {\rm and}\;\: z,
\end{array} \right.
\end{equation}
where
\beq{fr_sum}
\ffn_{\kkk} = \int_\vol\! d\rrr\: f(\rrr)\: e^{-i\kkk\cdot\rrr}.
\end{equation}
Note the prefactor $1/\vol$ in \eqref{fk_sum}. It is our choice to
put it there. Another choice would be to put it in \eqref{fr_sum},
or to put $1/\sqrt{\vol}$ in front of both equations. In all cases
the product of the normalization constants should be $1/\vol$.

An extremely important and very useful theorem states
\beq{delta_fct}
\int\!d\rrr\: e^{-i\kkk\cdot\rrr} = \vol\: \krondel{\kkk}{0},
\qquad \qquad
\frac{1}{\vol} \sum_{\kkk} e^{i\kkk\cdot\rrr} = \delta(\rrr).
\end{equation}
Note the dimensions in these two expressions so that you do not
forget where to put the factors of $\vol$ and $1/\vol$. Note also
that by using \eqref{delta_fct} you can prove that Fourier
transforming from $\rrr$ to $\kkk$ and then back brings you back
to the starting point: insert $\ffn_{\kkk}$ from \eqref{fr_sum} into
the expression for $f(\rrr)$ in \eqref{fk_sum} an reduce by use of
\eqref{delta_fct}.


\section{Continuous functions in an infinite region}
\seclab{InfiniteRegion}

If we let $\vol$ tend to infinity the $\kkk$-vectors become
quasi-continuous variables, and the $\kkk$-sum in \eqref{fk_sum} is
converted into an integral,
\beq{fk_sum_to_int}
f(\rrr) = \frac{1}{\vol}\sum_{\kkk} \ffn_{\kkk}\: e^{i\kkk\cdot\rrr}
\quad \mathop{\longrightarrow}_{\vol\rightarrow\infty} \quad
\frac{1}{\vol} \frac{\vol}{\twopicubed}
\int\!d\kkk\: \ffn_{\kkk}\: e^{i\kkk\cdot\rrr} =
\int\!\dktwopicubed\: \ffn_{\kkk}\: e^{i\kkk\cdot\rrr}.
\end{equation}
Now you see why we choose to put $1/\vol$ in front of $\sum_{\kkk}$.
We have
\beq{FourInf_rk} f(\rrr) = \int\!\dktwopicubed\:
\ffn_{\kkk}\: e^{i\kkk\cdot\rrr}, \qquad \qquad \ffn_{\kkk} =
\int\!d\rrr\: f(\rrr) e^{-i\kkk\cdot\rrr},
\end{equation}
and also
\beq{delta_rk}
\int\!\dktwopicubed\: e^{i\kkk\cdot\rrr} = \delta(\rrr),
\qquad \qquad
\int\!d\rrr\: e^{-i\kkk\cdot\rrr} = \twopicubed\: \delta(\kkk).
\end{equation}
Note that the dimensions are okay. Again it is easy to use these
expression to verify that Fourier transforming twice brings you back
to the starting point.

\section{Time and frequency Fourier transforms}
\seclab{time_freq_FT}

The time $t$ and frequency $\omega$ transforms can be thought of as an
extension of functions periodic with the finite period $\cal T$, to
the case where this period tends to infinity. Thus $t$ plays the role
of $\rrr$ and $\omega$ that of $\kkk$, and in complete analogy with
\eqref{FourInf_rk} -- but with the opposite sign of $i$ due to
\eqref{PlaneWave} -- we have
\beq{FourInf_tw}
f(t) =
\int_{-\infty}^{\infty}\!\domegatwopi\: \ffn_\omega\: e^{-i\omega t},
\qquad \qquad
\ffn_\omega = \int_{-\infty}^{\infty}\!dt\: f(t) e^{i\omega t},
\end{equation}
and also
\beq{delta_tw}
\int_{-\infty}^{\infty}\!\domegatwopi\: e^{-i\omega t} = \delta(t),
\qquad \qquad
\int_{-\infty}^{\infty}\!dt\: e^{i\omega t} = 2\pi\: \delta(\omega).
\end{equation}
Note again that the dimensions are okay.
